% !TEX root = monografia.tex

\documentclass[
	% -- opções da classe memoir --
	12pt,				% tamanho da fonte
	openright,			% capítulos começam em pág ímpar (insere página vazia caso preciso)
	oneside,			% para impressão em verso e anverso. Oposto a oneside
	letterpaper,		% tamanho do papel.
	% -- opções da classe abntex2 --
	%chapter=TITLE,		% títulos de capítulos convertidos em letras maíusculas
	%section=TITLE,		% títulos de seções convertidos em letras maíusculas
	%subsection=TITLE,	% títulos de subseções convertidos em letras maíusculas
	%subsubsection=TITLE,% títulos de subsubseções convertidos em letras maíusculas
	% -- opções do pacote babel --
	english,			% idioma adicional para hifenização
	%french,			% idioma adicional para hifenização
	%spanish,			% idioma adicional para hifenização
	%brazil,				% o último idioma é o principal do documento
	sumario=tradicional,
	]{abntex2}
	
% ---
% Pacotes fundamentais
% ---
\usepackage{cmap}				% Mapear caracteres especiais no PDF
\usepackage{lmodern}			% Usa a fonte Latin Modern		
\usepackage[T1]{fontenc}		% Selecao de codigos de fonte.
\usepackage[utf8]{inputenc}	
\usepackage{lastpage}			% Usado pela Ficha catalográfica
\usepackage{indentfirst}		
\usepackage{color}				
\usepackage[pdftex]{graphicx}	
\usepackage{epstopdf}      
\usepackage{amsfonts}  
\usepackage{subfig}
% ---
		
% ---
% ---
\usepackage{nomencl}
\usepackage{amsmath}
\usepackage{bbm}
\usepackage[chapter]{algorithm}
\usepackage{algorithmic}
\usepackage{multirow}
\usepackage{rotating}
\usepackage{pdfpages}

\usepackage{gensymb}
% ---

% ---
% ---

\usepackage[alf,abnt-etal-cite=2,abnt-etal-list=0,abnt-etal-text=emph]{abntex2cite}	% 

% ---

% ---
\usepackage{unasp}


% ---

% ---

% ---


\newcommand{\mb}[1]{\mathbf{#1}}

\newtheorem{mydef}{Definição}[chapter]
\newtheorem{lemm}{Lema}[chapter]
\newtheorem{theorem}{Teorema}[chapter]
\floatname{algorithm}{Pseudoc\'{o}digo}
\renewcommand{\listalgorithmname}{Lista de Pseudoc\'{o}digos}


% alterando o aspecto da cor azul
\definecolor{blue}{RGB}{41,5,195}


\makeatletter
\hypersetup{
     	%pagebackref=true,
		pdftitle={\@title},
		pdfauthor={\@author},
    	pdfsubject={\imprimirpreambulo},
	    pdfcreator={LaTeX with abnTeX2},
		pdfkeywords={abnt}{latex}{abntex}{abntex2}{trabalho acad\^{e}mico},
		%hidelinks,					% desabilita as bordas dos links
		colorlinks=false,       	    % false: boxed links; true: colored links		
    	linkcolor=red,          	% color of internal links
    	citecolor=blue,        		% color of links to bibliography
    	filecolor=magenta,      	% color of file links
		urlcolor=blue,
%		linkbordercolor={1 1 1},	% set to white
		bookmarksdepth=4
}
\makeatother
% ---

% ---
% ---

% O tamanho do parágrafo é dado por:
\setlength{\parindent}{1.3cm}

% Controle do espaçamento entre um parágrafo e outro:
\setlength{\parskip}{0.2cm}  % tente também \onelineskip

% ---
% Informacoes de dados para CAPA e FOLHA DE ROSTO
% ---
\titulo{INFOBEACONS}
\autor{
GABRIEL TAGLIARI RODRIGO
\par
MATHEUS SILVA HOFART
\par
ELISA BATTAGLINI DOMINGUES PEREIRA 
}
\local{ENGENHEIRO COELHO}
\data{2016}
\orientador{Roberto Sussumu Wataya}
%\coorientador[Co-orientador]{Prof. Dr. Co-orientador}
\instituicao{%
CENTRO UNIVERSITÁRIO ADVENTISTA DE SÃO PAULO
\par
CAMPUS ENGENHEIRO COELHO
\par
CURSO DE SISTEMAS PARA INTERNET
}
\tipotrabalho{Trabaho de Conclusão de Curso (TCC)}
\preambulo{Trabalho de Conclusão de Curso do Centro Universitário Adventista de São Paulo do curso de
Sistemas para Internet, sob a orientação do Prof. Dr. \imprimirorientador.}
% --- 
