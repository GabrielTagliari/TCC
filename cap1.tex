% !TEX root = monografia.tex
  \chapter{Introdução}
 
O tema deste projeto é a acessibilidade em museus através de \textit{beacons} e foi desenvolvido um aplicativo para \textit{smartphones} e \textit{tablets} que recebe informações dos \textit{beacons} e exibe informações na tela. O tema veio de uma ideia em conjunto, inspirada em projetos já existentes e utilizados em museus.

Quando vamos a um museu, é comum não encontrarmos todas as informações referentes ao que é exposto. Os visitantes querem ver e saber sobre aquilo que já fez parte da história através das esculturas, buscam entender o sentimento que os pintores querem passar através de suas pinturas, então se faz necessário criar uma forma de atender a essa procura. Além dessa dificuldade de informações existem aqueles que possuem uma barreira maior, os deficientes visuais, que utilizam, neste caso, os sentidos do tato e da audição. Além do toque, ainda sim precisam de outras maneiras de obterem informações sobre as exposições. 

Este estudo tem como objetivo criar um aplicativo baseado na tecnologia \textit{beacon}, que aprimora o acesso às informações dos objetos expostos no museu, através dos dispositivos móveis por pessoas deficientes visuais ou não. 

A importância deste trabalho se reflete em possibilitar aos visitantes do museu acesso as informações necessárias de uma forma rápida e dinâmica, melhorando a interação entre o museu e os usuários do aplicativo Infobeacons, trazendo consigo a inserção da acessibilidade, atributo essencial para locais que recebem visitantes com necessidades especiais ou que pretendem fazer essa inclusão.

Este aplicativo utiliza a tecnologia \textit{beacons}, um dispositivo que transmite sinais \textit{bluetooth}, esses sinais podem ser captados por aplicativos de \textit{smartphones} e \textit{tablets}. São colocados \textit{beacons} em cada obra que é exposta no museu, para que ao se aproximar com um \textit{smartphone} ou \textit{tablet} que possua este aplicativo instalado, será apresentada na tela e convertida em voz a informação cadastrada para aquele determinado \textit{beacon} no banco de dados. Consequentemente, ao se aproximar de outro \textit{beacon} o processo se repetirá com as informações referentes a ele. 
