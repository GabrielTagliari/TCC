% !TEX root = monografia.tex
\chapter{Fundamentação Teórica}

\section{Acessibilidade}
A acessibilidade envolve não só o meio físico mas também o meio digital, que possibilita a inclusão de portadores de deficiência em tudo que é oferecido para a sociedade. As informações são acessíveis a partir do momento em que são divulgadas de formas variadas como, por imagem, texto e som. \cite{torres2002acessibilidade}

\section{\textit{Java}}
A linguagem \textit{Java} surgiu a partir de uma pesquisa financiada pela \textit{Sun Microsystems} em 1991, iniciada sob o codinome de projeto \textit{Green}. Com o intuito de se tornar uma linguagem para dispositivos inteligentes destinados ao usuário final e com a premissa de ser uma linguagem próxima as linguagens \textit{C} e\textit{ C++} e que pudesse ser executado em diversos hardwares. \cite{java}

\section{\textit{Spring}}
A história do \textit{Spring} começou com um desenvolvedor em busca de qualidade no desenvolvimento Java para web. Esta pessoa, hoje um grande nome da comunidade Java, é Rod Johnson. Suas contribuições começaram com o livro chamado ‘’Expert One-on-One J2EE Design and Development’’. Nele, Rod mostra como utilizar as tecnologias \textit{J2EE} (atualmente \textit{Java EE}) para reduzir a complexidade do desenvolvimento, auxiliando ainda a resolver problemas e erros comuns. \cite{spring}

\section{\textit{Hibernate}}
O \textit{Hibernate} é um framework para o mapeamento objeto-relacional escrito na linguagem Java, mas também é disponível em .Net como o nome \textit{NHibernate}. Este programa facilita o mapeamento dos atributos entre uma base tradicional de dados relacionais e o modelo objeto de uma aplicação, mediante o uso de arquivos (\textit{XML}) para estabelecer esta relação. \cite{hibernate}

\section{\textit{HTML5}}
A sigla \textit{HTML} significa \textit{HyperText Markup Language} em português, linguagem de marcação de hipertexto. A primeira versão do \textit{HTML} foi baseada na linguagem \textit{SGML}. O SGML era utilizado para a estruturação de documentos e foi dele que o \textit{HTML} herdou diversas \textit{tags}, tais como: título <h1> ao <h6>, cabeçalho <head> e parágrafo <p>. A maior diferença entre essas duas linguagens de marcação é que o \textit{HTML} implementava a \textit{tag} <a> com o atributo \textit{href}, permitindo assim a ligação (links) de uma página a outra. Esse conceito de interligação entre documentos é a base do funcionamento de toda Web. \cite{html}

Para a criação do sistema de cadastramento foi usado o \textit{HTML5}, uma recente evolução do \textit{HTML} e uma nova versão da linguagem, com novos elementos, atributos e comportamentos.

\section{\textit{CSS3}}
O \textit{Cascading Style Sheets (CSS)} foi proposto pela primeira vez em outubro de 1994, por Hakon Lie, que queria facilitar a programação de sites, que na época era muito mais complexa. As pessoas tinham que utilizar mais códigos para chegar a um resultado simples, como criar uma tabela.
Em 1995 o \textit{CSS1} foi desenvolvido pela W3C, um grupo de empresas do ramo da informática. \cite{css}

Segundo Pacievitch, a linguagem de estilos ganhou muito destaque entre 1997 e 1999, neste período ficou conhecido por grande parte dos programadores.

\section{\textit{Bootstrap}}
Em 2011, o \textit{Bootstrap} foi criado como uma solução interna para resolver inconsistências de desenvolvimento dentro da equipe de engenharia do \textit{Twitter}. 
Embora inicialmente uma solução interna no \textit{Twitter}, Mark e Jacob rapidamente perceberam que tinham descoberto algum muito maior. Em agosto de 2011, a estrutura \textit{Bootstrap} foi lançada como um projeto de software livre no \textit{Github}. Em alguns meses, milhares de desenvolvedores de todo o mundo contribuíram com o código e o \textit{Bootstrap} se tornou o projeto de desenvolvimento de software livre mais ativo do mundo. \cite{bootstrap}

\section{\textit{Volley}}
O \textit{Volley} é uma biblioteca mantida pelo Google com a proposta de tornar a implementação de comunicação\textit{ HTTP} em aplicações \textit{Android} mais fácil e otimizada. Essa biblioteca também permite reduzir a quantidade de \textit{Boilerplate Code} em relação a implementações baseadas nas classes \textit{HTTP} nativas do\textit{ Android}, tornando o código mais simples e legível. \cite{volley}

\section{\textit{MySQL}}
O \textit{MySQL} é um servidor robusto de bancos de dados \textit{SQL} (\textit{Structured Query Language)} - Linguagem Estruturada para Pesquisas) muito rápido, multitarefa e multiusuário. O Servidor \textit{MySQL} pode ser utilizado em sistemas de produção com alta carga e missão crítica como também pode ser embutido em programa de uso em massa. \cite{mysql}
