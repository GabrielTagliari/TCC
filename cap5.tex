\chapter*{Considerações finais}
\label{concl}
\addcontentsline{toc}{chapter}{Conclusões}

Neste projeto abordamos sobre como resolver o problema de obter informações dentro de um museu pelos visitantes em geral, porém com atenção para aqueles que possuem deficiência visual.

Um dos objetivos iniciais do projeto era não utilizar internet para que os usuários pudessem utilizar o aplicativo em qualquer local. Contudo, foi decidido incorporar um sistema de cadastramento e um \textit{web service} para que a manutenção e o cadastramento de \textit{beacons} do projeto fosse realizada facilmente. Outro ponto que levou a esta tomada de decisão foi o fácil acesso atual a planos de 3g, logo, não haveria problema em utilizar dados móveis.

Consideramos que tudo que foi apresentado neste trabalho faz com que este problema encontrado seja sanado.

Este projeto foi de grande valor, visto que nos permitiu desenvolver algo que colaborasse tanto para resolver a dificuldade de informação quanto no quesito acessibilidade que leva em consideração a inclusão de portadores de deficiência visual em locais despreparados.

\section*{Perspectivas Futuras}

\begin{itemize}
    \item Hospedagem do \textit{web service}
    \item Publicar o aplicativo na \textit{Google Play}
\end{itemize}