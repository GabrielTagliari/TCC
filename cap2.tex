% !TEX root = monografia.tex
\chapter{Metodologia}

A metodologia utilizada neste estudo consistiu em quatro etapas, iniciando com a fundamentação teórica, de trabalhos relacionados e estado da arte. Na segunda etapa - fundamentação teórica, foram feitos conceitos sobre acessibilidade às informações pelas pessoas com deficiência; Na terceira etapa, estudos técnicos do \textit{beacons}, \textit{Android}, \textit{bluetooth}, \textit{talkback}. Na etapa seguinte a criação do aplicativo com a tecnologia \textit{beacons}, a criação do sistema de cadastramento de \textit{beacons}, e finalmente as considerações finais.

As ferramentas que foram usadas para todo o desenvolvimento foram: \textit{Android Studio}, \textit{Eclipse}, \textit{Google Chrome}, \textit{Opera}, \textit{MySQL WorkBench}, \textit{Brackets} e \textit{Photoshop}, \textit{WampServer}.

Foi utilizado um computador \textit{desktop} com processador \textit{AMD FX 8350}, com sistema operacional Windows 8.1, 6Gb de memória ram e um notebook \textit{HP G42} com processador \textit{Intel Core i5}, sistema operacional \textit{Ubuntu} 15.10 e 2Gb de memória ram.

Após a realização do desenvolvimento iniciou-se a parte dos testes para garantir a funcionalidade tanto do aplicativo quanto do sistema de cadastramento.